\documentclass[12pt, %
openright, 
oneside, %
%twoside, %TCC: Se seu texto tem mais de 100 páginas, descomente esta linha e comente a anterior
a4paper,    %
%english,   %
brazil]{facom-ufu-abntex2}


\usepackage{graphicx}
\graphicspath{{figuras/}{pictures/}{images/}{./}} % where to search for the images

\newcommand{\blue}[1]{\textcolor{blue}{#1}}
\newcommand{\red}[1]{\textcolor{red}{#1}}


\autor{José Lucas Ferreira de Lima} %TCC
\data{2025}
\orientador{Ronaldo Castro de Oliveira} %TCC
%\coorientador{Algum?} %TCC

% ---
% Informações de dados para CAPA e FOLHA DE ROSTO
% ---

\titulo{De voltas rápidas a decisões rápidas: o papel do big data na Fórmula 1} %TCC

\hypersetup{pdfkeywords={Big Data, Fórmula 1, telemetria, análise de dados, Kafka, UDP, Python, dashboard, machine learning, simulação, mensageria}}

\begin{document} 
\frenchspacing 

% ----------------------------------------------------------
% ELEMENTOS PRÉ-TEXTUAIS
% ----------------------------------------------------------
%\pretextual
\imprimircapa
\imprimirfolhaderosto


% ---
% Inserir folha de aprovação
% ---
%
% \includepdf{folhadeaprovacao_final.pdf} %TCC: depois de aprovado o trabalho, descomente esta linha e comente o próximo bloco para incluir scan da folha de aprovação.
%
\begin{folhadeaprovacao}

  \begin{center}
    {\ABNTEXchapterfont\large\imprimirautor}

    \vspace*{\fill}\vspace*{\fill}
    {\ABNTEXchapterfont\bfseries\Large\imprimirtitulo}
    \vspace*{\fill}
    
    \hspace{.45\textwidth}
    \begin{minipage}{.5\textwidth}
        \imprimirpreambulo
    \end{minipage}%
    \vspace*{\fill}
   \end{center}
    
   Trabalho aprovado. \imprimirlocal, 01 de novembro de 2016: %TCC:

   \assinatura{\textbf{\imprimirorientador} \\ Orientador}  
   \assinatura{\textbf{Professor}}% \\ Convidado 1} %TCC:
   \assinatura{\textbf{Professor}}% \\ Convidado 2} %TCC:
   %\assinatura{\textbf{Professor} \\ Convidado 3}
   %\assinatura{\textbf{Professor} \\ Convidado 4}
      
   \begin{center}
    \vspace*{0.5cm}
    {\large\imprimirlocal}
    \par
    {\large\imprimirdata}
    \vspace*{1cm}
  \end{center}
  
\end{folhadeaprovacao}
% ---


%%As seções dedicatória, agradecimento e epígrafe não são obrigatórias.
%%Só as mantenha se achar pertinente.

% ---
% Dedicatória
% ---
\begin{dedicatoria}
   \vspace*{\fill}
   \centering
   \noindent
   \textit{Dedico este trabalho aos meus pais, pelo apoio incondicional em toda a minha trajetória; aos meus amigos, 
   que foram pilares essenciais durante minha jornada acadêmica; e à minha avó, que nos deixou enquanto eu construía este trabalho, 
   mas cuja memória permanece viva em meu coração.}  %TCC:
   \vspace*{\fill}
\end{dedicatoria}
% ---

% ---
% Agradecimentos
% ---
%\begin{agradecimentos}
%Agradeço a \lipsum[30]. %TCC:
%\end{agradecimentos}
% ---

% ---
% Epígrafe
% ---
%\begin{epigrafe}
%    \vspace*{\fill}
%	\begin{flushright}
%		\textit{``Alguma citação que ache conveniente? \lipsum[10]''} %TCC:
%	\end{flushright}
%\end{epigrafe}
% ---



\begin{resumo} %TCC:

  A Fórmula 1 é um ambiente altamente tecnológico onde a análise de dados em 
  tempo real desempenha um papel crucial no desempenho das equipes, deixando a competitividade do esporte
  cada vez mais alta. Este trabalho explora a aplicação de \textbf{Big Data} na Fórmula 1, abordando tanto os conceitos teóricos 
  quanto uma implementação prática baseada na telemetria do jogo \textbf{F1 22}.
  
  O objetivo principal é demonstrar como a coleta e análise de dados podem fornecer insights valiosos para 
  a tomada de decisões estratégicas durante uma corrida. Para isso, foi desenvolvida uma aplicação em \textbf{Python} 
  que se conecta ao jogo pela rede local, estabelece um \textbf{socket} com este, recebe os dados de telemetria via \textbf{UDP}, 
  processa as informações, e envia os dados por mensageria utilizando \textbf{Kafka}. Os resultados são exibidos em um 
  \textbf{dashboard interativo} conectado a essa fila, permitindo análises em tempo real durante a corrida. 
  A aplicação foi testada em um ambiente controlado, e os resultados obtidos são discutidos em detalhes.
  
  A metodologia adotada inclui o estudo dos princípios do \textbf{Big Data} aplicados à F1, bem como o desenvolvimento 
  da pipeline de dados para captura e visualização dos dados do jogo. A implementação prática busca simular a experiência 
  real das equipes de F1 no uso de dados para análise de desempenho, ou seja, a visão do engenheiro de corrida durante a prova.

  Os resultados esperados incluem a demonstração do impacto do Big Data na performance das equipes e a 
  validação do potencial da tecnologia utilizada para fins de simulação e aprendizado. Este trabalho contribui 
  para a compreensão do uso da telemetria na F1 e reforça a importância da análise de dados em cenários de alta performance.

 \vspace{\onelineskip}
    
 \noindent
 \textbf{Palavras-chave}: big data, formula 1, engenharia de dados, python, pipeline. %TCC:
\end{resumo}

% ---
% inserir lista de ilustrações
% ---
\pdfbookmark[0]{\listfigurename}{lof}
\listoffigures*
\cleardoublepage
% ---

% ---
% inserir lista de tabelas
% ---
\pdfbookmark[0]{\listtablename}{lot}
\listoftables*
\cleardoublepage
% ---



% ---
% inserir lista de abreviaturas e siglas
% ---
\begin{siglas} %TCC:
  \item[Fig.] Area of the $i^{th}$ component
  \item[456] Isto é um número
  \item[123] Isto é outro número
  \item[Zézão] este é o meu nome
\end{siglas}
% ---

%% ---
%% inserir lista de símbolos, se for adequado ao trabalho. %TCC:
%% ---
%\begin{simbolos}
%  \item[$ \Gamma $] Letra grega Gama
%  \item[$ \Lambda $] Lambda
%  \item[$ \zeta $] Letra grega minúscula zeta
%  \item[$ \in $] Pertence
%\end{simbolos}
%% ---

% ---
% inserir o sumario
% ---
\pdfbookmark[0]{\contentsname}{toc}
\tableofcontents*
\cleardoublepage
% ---





% ----------------------------------------------------------
% ELEMENTOS TEXTUAIS
% ----------------------------------------------------------
\textual


% ----------------------------------------------------------
% Introdução
% ----------------------------------------------------------

\chapter[Introdução]{Introdução}
%TCC:
Contextualização, problema, hipótese, objetivo geral, objetivos específicos, justificativa e resultados esperados.
\section{Contextualização e Motivação}
\section{Problema de Pesquisa}
\section{Objetivos}
\subsection{Objetivo Geral}
\subsection{Objetivos Específicos}
\section{Justificativa}
\section{Metodologia do Trabalho}
\section{Estrutura do Trabalho}

% ---
% Revisão Bibliográfica
% ---

\chapter{Revisão Bibliográfica}
%TCC:
\section{Conceitos de Big Data}
Big Data nada mais é do que a capacidade de coletar, processar e analisar grandes volumes de dados em alta velocidade,
variabilidade e veracidade. A definição dos 5 Vs é comumente utilizada para descrever as características do Big Data:

\begin{itemize}
    \item \textbf{Volume}: refere-se à quantidade de dados gerados a cada segundo, minuto ou hora. 
    \item \textbf{Velocidade}: refere-se à rapidez com que os dados são gerados e processados.
    \item \textbf{Variedade}: refere-se à diversidade de fontes e formatos dos dados.
    \item \textbf{Veracidade}: refere-se à confiabilidade e precisão dos dados.
    \item \textbf{Valor}: refere-se à capacidade de extrair informações úteis e insights dos dados.
\end{itemize}

Através do big data, é possível analisar padrões, tendências e correlações em tempo real, permitindo a tomada de decisões mais assertivas
e eficientes. É justamente com ela que as equipes de Fórmula 1 conseguem monitorar o desempenho dos carros em tempo real

Entretanto, exige um poder computacional significativo e ferramentas especializadas para processamento e análise dos dados,
bem como profissionais capacitados para interpretar os resultados e extrair valor das informações coletadas. Atualmente,
já existem diversas tecnologias e frameworks disponíveis para lidar com o Big Data, como o Apache Hadoop, Spark, Kafka, entre outros.
Algumas delas serão bem úteis e utilizadas ao longo deste trabalho.

\section{Aplicação de Big Data na Fórmula 1}
Atualmente, nos carros de Fórmula 1, contamos com 300 sensores espalhados pelo carro, que geram dados de temperatura do motor, pressão 
dos pneus, velocidade, aceleração, entre outros. Esses dados são coletados em tempo real e enviados para a equipe de engenheiros dentro do paddock,
que podem desempenhar diversas atividades com eles, desde a análise de desempenho do carro e tomada de decisões estratégicas durante a 
corrida, até a simulação de cenários e previsão de resultados.

Por exemplo, durante uma corrida, os engenheiros podem monitorar a temperatura dos pneus e ajustar a pressão de acordo com as 
condições da pista, ou analisar o desgaste dos freios e sugerir ao piloto uma redução no uso do freio motor. Além disso, é possível
usar esses dados já tratados pela nossa pipeline para passá-los para um modelo de machine learning, que pode prever o desempenho do carro,
se uma ultrapassagem é viável, ou até mesmo simular cenários diversos de corrida.

Alguns modelos já são fornecidos por empresas patrocinadoras, como a AWS, que oferece um modelo de machine learning para prever o 
desempenho do carro, além do algoritmo de ultrapassagem, e a Oracle, atual patrocinadora da Red Bull, que oferece um modelo de simulação
de corrida. Entretanto, esses modelos são genéricos e não levam em consideração as particularidades de cada carro e piloto, o que pode
comprometer a precisão das previsões.

E é justamente por isso que não coletamos apenas dados do carro, mas também do piloto, como o nível de concentração, batimentos cardíacos,
e até mesmo a pressão sanguínea. Esses dados são fundamentais para entender o comportamento do piloto durante a corrida e ajustar o carro
de acordo com suas necessidades, garantindo o melhor desempenho possível.

Além disso, não só as equipes podem tirar proveito do big data, como também a própria FIA, que pode, por exemplo, usar dados dos eventos
para melhorar a segurança dos pilotos, ou até mesmo prever acidentes antes que eles aconteçam. Outra empresa que se beneficia bastante desses
dados gerados é a Liberty Media, que detém os direitos comerciais da Fórmula 1 e pode usá-los para melhorar a experiência do espectador,
oferecendo insights e estatísticas em tempo real durante as corridas, entender o que pode ser melhorado nos próximos GPs para quem assiste
no autódromo. Não se pode esquecer também dos patrocinadores, que podem usar esses dados para entender melhor o retorno sobre o investimento
e ajustar suas estratégias de marketing.

\section{Telemetria e Coleta de Dados em Tempo Real}
O grande desafio de todas as equipes até aqui: como coletar gigabytes de dados de diversos formatos em uma rede privada, compartilhada apenas
entre as equipes, de forma segura, eficiente e no menor espaço de tempo possível, e mais, sem deixar esse dado ser corrompido ou perdido? 
A resposta para isso é a telemetria, que é a tecnologia que permite a coleta, transmissão e análise de dados em tempo real.

\section{Tecnologias Utilizadas}
\section{Trabalhos Relacionados}

% ---
% Desenvolvimento
% ---

\chapter{Desenvolvimento}
%TCC:
Um ou mais capítulos (por exemplo um para testes)
\section{Arquitetura da Solução}
\section{Coleta de Dados via Telemetria}
\section{Processamento e Armazenamento de Dados}
\section{Visualização dos Dados}
\section{Implementação e Testes}


\begin{figure}[!ht]
    \centering
	\includegraphics[width=0.55\linewidth]{imagemExemplo.pdf}
	\caption[Isso é o que aparece no sumário]{Imagem de exemplo.}
	\label{fig:graficosVariandoTamanhoRede}
\end{figure}

% ---
% Análise dos Resultados
% ---

\chapter{Análise dos Resultados}
\section{Validação da Coleta e Processamento}
\section{Desempenho e Eficiência da Aplicação}
\section{Limitações e Melhorias Futuras}




%TCC:
%TCC:
%TCC:
%TCC:

% ---
% Conclusão
% ---
\chapter{Conclusão e Trabalhos Futuros}
%TCC:
E daí?
\section{Revisão dos Objetivos e Contribuições}
\section{Desafios Encontrados}
\section{Aplicações Futuras}





% ----------------------------------------------------------
% ELEMENTOS PÓS-TEXTUAIS
% ----------------------------------------------------------
\postextual


% ----------------------------------------------------------
% Referências bibliográficas
% ----------------------------------------------------------
\bibliography{abntex2-modelo-references}


%% ----------------------------------------------------------
%% Apêndices TCC: só mantenha se for pertinente.
%% ----------------------------------------------------------

% ---
% Inicia os apêndices
% ---
\begin{apendicesenv}

% Imprime uma página indicando o início dos apêndices
\partapendices

% ----------------------------------------------------------
\chapter{Quisque libero justo}
% ----------------------------------------------------------

\lipsum[50]

% ----------------------------------------------------------
\chapter{Coisas que fiz e que achei interessante mas não tanto para entrar no corpo do texto}
% ----------------------------------------------------------
\lipsum[55-57]

\end{apendicesenv}
% ---


% ----------------------------------------------------------
% Anexos %TCC: so mantenha se pertinente.
% ----------------------------------------------------------

% ---
% Inicia os anexos
% ---
\begin{anexosenv}

% Imprime uma página indicando o início dos anexos
\partanexos

% ---
\chapter{Eu sempre quis aprender latim}
% ---
\lipsum[30]

% ---
\chapter{Coisas que eu não fiz mas que achei interessante o suficiente para colocar aqui}
% ---

\lipsum[31]

% ---
\chapter{Fusce facilisis lacinia dui}
% ---

\lipsum[32]

\end{anexosenv}

%---------------------------------------------------------------------
% INDICE REMISSIVO
%---------------------------------------------------------------------

\printindex



\end{document}